\documentclass[a4paper, 14pt]{article}
\usepackage[left=2cm,right=2cm,top=2cm,bottom=2cm]{geometry}
\usepackage{wrapfig} \usepackage{graphicx} \usepackage{mathtext}
\usepackage{amsmath} \usepackage{siunitx} % Required for alignment
\usepackage{subfigure} \usepackage{multirow} \usepackage{rotating}
\usepackage[T1,T2A]{fontenc} \usepackage[russian]{babel} \usepackage{caption}
\usepackage{hyperref}
\usepackage[]{float}
\title{Контрольные вопросы}
\begin{document}
\maketitle
\textbf {1) Граф} - это набор точек (вершин) и ребер (отрезков, соединящих две вершины). Обозначаются как $ G(E,V) $ \\

\textbf{2) Наибольшее количество ребер} M в неориентрованном графе - $ \dfrac{N\cdot(N-1)}{2•} $, так как каждая вершина не имеет связи с собой и имеет максимум одно ребро с каждой из вершин. В случае ориентированного графа ребро может соединять вершины в одном направлении, но не в обратном. Поэтому в случае ориентрованного графа N $ \leq N(N-1) $\\

\textbf{3)} Проверить направленность графа можно при помощи \textbf{матрицы смежности}. Если она будет симметрична-граф неориентированный .То есть если поменять строку и столбец местами (поменять местами две вершины), значение в матрице должно совпадать.\\

\textbf{4)} При добавлении веса каждому ребру в списке ребер вместо пары чисел должна вводится тройка. В списке смежностей помимо каждой вершины должен вводится вес ребра, ведущего к этой вершине.\\

\textbf{5) Компонента связности}- это такая часть графа, между любыми двумя вершинами которой есть связь (через одно ребро либо через несколько вершин, входящих в компоненту).\\

\textbf{6)} При помощи BFS можно искать циклы в графе. Если нужно найти кратчайший цикл, то выгоднее будет использовать BFS, в случае, если нужно найти любой цикл, будет выгоднее использовать DFS.\\

\textbf{7)} Алгоритм Дейкстра не работает с отрицательными ребрами, так как не будет учитываться возможность найти более короткий путь к вершине. Решить это можно прибавив какое-нибудь большое число к отрицательным ребрам и вычесть его из итоговой длины столько раз, сколько отрицательных ребер встретилось на пути к этой вершине.\\

\textbf{8)} Алгоритм Форд Беллмана работает N-1 раз, чтобы найти проверить все пути и запускается N-й раз для проверки отриуательных циклов, если расстояние меняется, то в этом случае искать кратчайшее расстояние будет некорректно, так как есть отрицательный цикл, прроходя по коттрому расстояние будет уменьшаться сколько угодно раз\\

\textbf{9}) Алгоритм Форд-Беллмана будет работать быстрее только в случае маленького числа M или сильно разреженного графа. Тогда асимптотика будет $O(N) $ и так как операци более простые, константа будет меньше. \\

\textbf{10)} В случае, если глубина рекурсии будетт слишком большая, стек может быть переполнен и будет выгоднее использовать итеративную реализацию DFS. В случае не слишком большой глубины выгоднее использовать рекурсию. Использование списка как стека будет эффективно, так как добавление и удаление элемента происходит за О(1). В случае BFS использовать очередь будет неэффективно так как удаление из начала очереди будет происходить за О(1), поэтому лучше использовать стек.  
\clearpage
\section{Задания}

1.\begin{verbatim}
def edge_list():
    N = int(input())
    M = int(input())
    res = []
    for i in range(M):
        x, y = (int(x) for x in input().split())
        res.append((x, y))
    return N, res
\end{verbatim}
2. \begin{verbatim}
def adj_matrix(N, edge_list):
    res = [[0 for j in range(N)] for i in range(N)]

    for x, y in edge_list:
        res[x][y] = 1
    return res
\end{verbatim}
3. \begin{verbatim}
def reversed_adj(matrix):
    N = len(matrix)
    reversed_matrix = [[0] * N for _ in range(N)]
    for i in range(N):
        for j in range(N):
            reversed_matrix[j][i] = matrix[i][j]

    return reversed_matrix
\end{verbatim}
4.\begin{verbatim}
def components(adj_list):
    def BFS(start_node, visited):
        queue = [start_node]
        visited.add(start_node)
        while queue:
            curr_node = queue.pop(0)
            for adj_node in adj_list.get(curr_node, []):
                if adj_node not in visited:
                    visited.add(adj_node)
                    queue.append(adj_node)
    visited = set()
    count = 0
    for node in adj_list:
        if node not in visited:
            BFS(node, visited)
            count += 1
    return count
\end{verbatim}
\clearpage
5.\begin{verbatim}
def DFS(curr_node, adj_list, visited=None):
    if visited is None:
        visited = set()

    print(curr_node)
    visited.add(curr_node)

    for adj_node in adj_list[curr_node]:
        if adj_node not in visited:
            DFS(adj_node, adj_list, visited)
    print(curr_node)
    return visited
\end{verbatim}
6.\begin{verbatim}
def FB(curr_node, adj_list):
    d = [float("inf")] * len(adj_list)
    d[curr_node] = 0
    for i in range(len(adj_list) - 1):
        for x in range(len(adj_list)):
            for y in adj_list[x]:
                if d[x] != float("inf") and d[x] + 1 < d[y]:
                    d[y] = d[x] + 1
    for x in range(len(adj_list)):
        for y in adj_list[x]:
            if d[x] != float("inf") and d[x] + 1 < d[y]:
                print("Отрицательный цикл")

    return d
\end{verbatim}
\end{document}